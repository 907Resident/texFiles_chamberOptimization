\section{Overview}
	\subsection{Introduction}
	The Earth remains a habitable planet largely in part to the greenhouse gases (GHG) such as methane (\ce{CH4}) and carbon dioxide (\ce{CO2}) that insulate the Earth by absorbing and emitting infrared radiation. Properly accounting for the pathways of GHGs into the atmosphere is essential to mitigating climate warming.
	
	\vspace{3mm}  \par \noindent
	Studies and long-term monitoring for changes in \ce{CO2} have been heavily documented for decades \citep{Keeling1960, Friedlingstein2014}. However, \ce{CH4}, is also an important GHG; one molecule of \ce{CH4} traps 86 times more thermal energy than a molecule of \ce{CO2} over a 20-year period and there are even claims that estimate is too low \citep{Howarth2015, Bridgham2013}. This ratio, also known as global warming potential has stimulated research during the last decade or so on the routes \ce{CH4} takes to enter the atmosphere. Though human-derived, or anthropogenic, sources are important to quantify, natural reservoirs of \ce{CH4} contribute about 51\% of total emissions in the previous decade \citep{Kirschke2013}. But estimates for natural sources currently contain a greater degree of uncertainty than their anthropogenic counterparts \citep{Kirschke2013}. 
	
	\vspace{3mm}  \par \noindent 
	Wetlands are the dominant source of natural production of \ce{CH4}, but they alone only account for about 60\% of natural emission, leaving the remaining 40\% to microbial activity in the soil microbes, termites, and even the deep ocean \citep{Kirschke2013, Etiope2002a}. Another important natural source of \ce{CH4} are derived from ongoing geologic processes, otherwise known as geogenic. Currently, geogenic emissions are estimated to produce 30-70 Mt \ce{CH4} yr$^{-1}$, which is on par with other non-wetland natural sources \citep{Etiope2002a}. This estimate was calculated by measuring \ce{CH4} from active points of emission (e.g. vents and fumaroles) and diffuse areas (thermally altered earth).
	
	
	\vspace{3mm}  \par \noindent 
	However, previous estimates are only from European locations, thus stimulating a need for diffuse \ce{CH4} measurements in N. America. This research seeks to further constrain estimates of \ce{CH4} emissions by providing the first geologic measurements of diffuse \ce{CH4} in N. American volcanic caldera and hydrothermal environments, specifically, Valles caldera (VC) and Yellowstone caldera (YC). Furthermore, collection of \ce{CH4} and \ce{CO2} emissions from numerous sites within VC and YC will permit the characterization of spatial and temporal variability; upon integrating that variability we will be able to: \textbf{estimate total emissions, assess correlations between emissions and/or geyser activity with time and space to constrain subsurface transport mechanisms, and lastly, make overall comparisons between the two calderas to highlight controls between emission rates}.
	
	\subsection{Site Description}
	A collapsed caldera is a large, basin-shaped depression formed by the collapse of the roof of a volcano following a large eruption. Valles Caldera is located in northcentral New Mexico and is one of several features within the Jemez Mountain Volcanic Field. VC initially erupted 1.2 million years ago, creating a caldera 20 km in diameter \citep{Goff2009}. Yellowstone Caldera is located in northwestern Wyoming and is a product of the North American plate moving west across a stationary mantle hotspot, causing linear chain of volcanic eruptions \citep{Smith1994}. An eruption 640 thousand years (kya) created YC and left behind a caldera that is roughly 64 km across. The most recent eruption at VC was 40 kya and for YC, 70 kya; these recent eruptions coupled with active hydrothermal systems reveal that both calderas are supplied with anonymously high subsurface heat sources. 
	
	\vspace{3mm}  \par \noindent	
	However, one important difference between the sites is that YC is more active than VC, which is exemplified with the greater abundance of thermal features, higher surface heat flow, and larger and more frequent earthquakes \citep{Lowenstern2008}. Together, VC and YC are two of the three major Quaternary calderas in the US and both are considered active volcanoes, making them relevant candidates for quantifying the contribution of geologic CH4 into the atmosphere.
	
	\subsection{Research Design \& Data Collection Methods}
	Emissions will be measured by capturing \ce{CH4} and \ce{CO2} as it migrates from the subsurface toward the atmosphere using a chamber sealed tightly to the surface. This Eosense autochamber (eosAC) will be connected to a Picarro G2201-i Cavity Ring-Down Spectrometer (CRDS) that can measure both gases rapidly (4 Hz) and in situ for concentrations ([\ce{CH4}] and [\ce{CO2}]) and carbon isotopic composition ($\delta^{13}$C-\ce{CH4} and $\delta^{13}$C-\ce{CO2}). Over several years, I have created a successful methodology that I will continue to use for these measurements. The concentration data will be necessary for making emission estimates and the carbon isotopic composition will be essential for characterizing the source of the gases and the temperature at which these gases formed. 
	
	\vspace{3mm}  \par \noindent 	
	Research shows the effectiveness of the CRDS-eosAC method \citep{Christiansen2015a}, but mobility is limited due to weight of the equipment ($> 54$ kg). For sites that I cannot access with the CRDS-eosAC method, I will use static PVC chambers \citep{Livingston2006}. Gas samples are withdrawn from the chamber headspace with a syringe and injected into a vial. \ce{CH4} concentration and $\delta^{13}$C-\ce{CH4}, and $\delta^{2}$H-\ce{CH4} will be shipped to UC Davis for GC-IRMS analysis. CO2 concentration and $\delta^{13}$C-\ce{CO2} will be measured using a GC-MS at the Center for Stable Isotopes (Univ. of New Mexico) with our collaborator, Prof. Tobias Fischer. 